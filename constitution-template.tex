\documentclass[12pt]{article}

% Packages
\usepackage{fullpage}
\usepackage{enumerate}
\usepackage{titlesec}
\usepackage{appendix}

% Settings
\newcommand{\orgname}{NAME OF ORGANIZATION\space}
\newcommand{\orgemail}{test@example.com}
\newcommand{\creationdate}{March 14, 2014}
\titleformat{\section}{\normalfont\large\bfseries}{\thesection.}{0.5em}{}


\begin{document}

%%%%%%%%%%%%%%%%%%%% TITLE PAGE %%%%%%%%%%%%%%%%%%%%

\begin{titlepage}
\begin{center}

% School Name
\textsc{\LARGE University of Toronto}\\[3.0cm]

% Title
{ \huge \bfseries \orgname Constitution\\[1.0cm] }

% Executives and Supervisor
\begin{minipage}{0.4\textwidth}
\begin{flushleft} \large
\emph{Executive:}\\
First \textsc{Last}
\end{flushleft}
\end{minipage}
\begin{minipage}{0.4\textwidth}
\begin{flushright} \large
\emph{Supervisor:} \\
First \textsc{Last}
\end{flushright}
\end{minipage}

\vfill

% Email and Date
{\orgemail}\\
{\creationdate}
\end{center}
\end{titlepage}


%%%%%%%%%%%%%%%%%%%% ARTICLES %%%%%%%%%%%%%%%%%%%%
\part{Articles}

% Article 1
\section{Name of Organization}
\begin{enumerate}[{1}.1]
    \item The official name of the organization will be \orgname at the University of Toronto.
    \item The \orgname may be referred to by the acronym HH.
\end{enumerate}


% Article 2
\section{Purpose}
\begin{enumerate}[{2}.1]
    \item	The purpose of \orgname will be to educate members about technology.
    \item	The \orgname fundamentally serves a non-profit function within the University of Toronto, and will not engage in activities that are essentially commercial in nature. 
    \item	The \orgname operates as an independent entity working within the University of Toronto community subject to the values and policies of the University.  
\end{enumerate}


% Article 3
\section{Membership}
\begin{enumerate}[{3}.1]
    \item Membership in \orgname is open to all students, staff, faculty and alumni of the University of Toronto.
    \item The term of membership for the \orgname will be from September 1 – August 31 each year.
    \item Each member shall be afforded the following rights through membership in \orgname: 
    \begin{enumerate}[{3.3}.1]
        \item	The right to participate and vote in group elections and meetings;
        \item	The right to communicate and to discuss and explore all ideas;
        \item	The right to organize/engage in activities/events that are reasonable and lawful;
        \item	The right to freedom from discrimination on the basis of sex, race, religion, or sexual orientation;
        \item	The right to be free from censorship, control, or interference by the University on the basis of the organization’s philosophy, beliefs, interests or opinions unless and until these lead to activities which are illegal or which infringe on the rights and freedoms already mentioned above;
        \item	The right to distribute on campus, in a responsible way, published material provided that it is not unlawful;
    \end{enumerate}
    \item Each member shall possess the following responsibilities relative to participation in \orgname: 
    \begin{enumerate}[{3.4}.1]
        \item	Support the purpose of the organization; 
        \item	Uphold the values of the organization;
        \item	Contribute constructively to the programs and activities offered by the organization; 
        \item	Attend general meetings; 
        \item	Abide by the constitution and subsequent official organizational documents;
        \item	Respect the rights of peers and fellow members;
        \item	Abide by University of Toronto policies, procedures, and guidelines;
        \item	Abide by the Laws of the Land, including but not limited to the Criminal Code of Canada. 
    \end{enumerate}
    \item The \orgname will collect a mandatory membership fee from each member each year. This fee will proposed as part of the operating budget presented to general members for approval at a valid general meeting. 
    \item The \orgname values and respects the personal information of its members. The \orgname secures its member’s information at all times and will not supply names or other confidential information to third-parties. 
    \item The \orgname will protect the privacy of member information and must use it only for the delivery of service and not for commercial gain.
\end{enumerate}


% Article 4
\section{Executive}
\begin{enumerate}[{4}.1]
    \item The executives of the organization shall include one (1) President and two (2) Vice-Presidents.
    \item The broad responsibilities of each executive position are as follows:
    \begin{enumerate}[{4.2}.1]
        \item	President is the official spokesperson of the organization and provides direction for all components of the organization in a manner consistent with the organization’s constitution and policies.
        \item	Vice-Presidents are the official secondary spokespeople of the organization and provides support for all components of the organization in a manner consistent with the organization’s constitution and policies.
    \end{enumerate}
    \item Only student members of the organization may hold executive positions. 
    \item The executive positions collectively will form a committee that acts as the primary steward of the organization.
    \item This committee is collectively responsible for the day-to-day decision making of the organization including but not limited to monitoring finances, event planning and execution, member services, and advocating on behalf of members to Administration and student government. 
    \item This committee cannot make amendments to the constitution without the approval of the general membership at a valid general meeting. 
    \item The term of each executive will last from May 1 following their election to April 30 of the following year.
    \item Any executive of the organization may resign, provided that such resignation is made in writing and delivered to the President. Unless any such resignation is, by its terms, effective on a later date, it shall be effective on delivery to the President, and no ratification by the organization shall be required to make the resignation official.
    \item Any vacancy of executives shall be filled by the President or designate of the organization until such a time where a by-election is held, a permanent appointment occurs, or a hiring process is conducted. 
    \item	If the President resigns, notice of such resignation must be submitted in writing and delivered to the executive committee at a valid executive meeting. Unless any such resignation is, by its terms, effective on a later date, it shall be effective on delivery to the executive committee, and no ratification by the organization shall be required to make the resignation official.
    \item Any vacancy of the President shall be filled by another executive committee member appointed by a simple and clear majority of the executive committee until such a time where a by-election is held, a permanent appointment occurs, or a hiring process is conducted.
\end{enumerate}


% Article 5
\section{Removal of Members and Executives}
\begin{enumerate}[{5}.1]
    \item The process for removing a member or executive may be initiated when a committee of no less than three (3) non-executive general members and two (2) executives appointed by the general membership to investigate a complaint determines that:
    \begin{enumerate}[{5.1}.1]
        \item	A member or executive has engaged in unlawful actions or activities;
        \item	A member or executive has violated the constitution;
        \item	A member or executive has violated University of Toronto policies, procedures, or guidelines;
        \item	A member or executive has violated the rights of a fellow member;
        \item	A member or executive has not fulfilled their organizational responsibilities;
        \item	Other criteria deemed to be appropriate by the Executive Committee in consultation with and approved by a majority of the general membership. 
    \end{enumerate}
    \item The process for removing a member or executive may also be initiated when:
    \begin{enumerate}[{5.2}.1]
        \item	A petition calling for a vote and bearing the signatures of a majority of the general membership is submitted to any member of the executive.
        \item A motion for a removal vote is put forward by any member of the executive and passed by a two-thirds majority vote of the executives.  The individual facing potential removal vote is entitled to vote on the motion if they are an executive or be given an opportunity to explain themselves if they are a non-executive general member.
    \end{enumerate}
    \item The removal of members and executives will be facilitated by a three tier procedure which operates as follows: 
    \begin{enumerate}[{5.3}.1]
        \item First Tier:
        \begin{itemize}
            \item The executive or member will be warned both verbally and in writing that their behavior constitutes grounds for removal from the organization and that it should cease effective immediately. 
        \end{itemize}
        \item	Second Tier: 
        \begin{itemize}
            \item	Initiated because the member or executive has violated section 5.1 after receiving a first tier warning relative to a particular action or behavior.
            \item An executive belonging to the committee will be appointed as the Mediator by the committee.
            \item	The Mediator will be responsible for contacting the executive or member and facilitating training or suggesting best practices on how to correct the issues of concern. 
            \item	The Mediator must address all complaints in writing by formulating an action plan and timeline to correct any issues involving executives or members within fourteen (14) calendar days. 
            \item	The executive or member accused of violating section 5.1 will be given fourteen (14) calendar days from receiving the Mediator's written response to demonstrate progress or correction of behavior. 
        \end{itemize}
        \item Third tier:
        \begin{itemize}
            \item	Initiated because the member or executive has violated section 5.1 after receiving second tier warning relative to a particular action or behavior.  
            \item	The removal vote must take place at a valid general meeting of the membership.  A representative supporting the motion for removal and the executive or member facing removal (or an individual they designate), may speak for up to five minutes each.
            \item	The removal of an executive or member requires a 2/3 majority vote of all of the members present at a valid general meeting (including executives).  The executive or member facing removal is entitled to vote on the motion. 
        \end{itemize}
    \end{enumerate}
\end{enumerate}


% Article 6
\section{Finances}
\begin{enumerate}[{6}.1]
    \item The funds of the organization shall be expended pursuant to the operating budget approved by the general membership at a valid general meeting. 
    \item Notwithstanding section 6.1, the Executive Committee may not approve any unbudgeted expenditure of the organization’s funds above \$100.00 without the approval of the general members at a valid general meeting. 
    \item All Budgets shall be prepared by the executives in accordance with the organization’s priorities as determined by the executive committee in consultation with general members at a valid general meeting. 
    \item The Executive Committee shall present a proposed operating budget for the next fiscal year to the general membership for its consideration at the final general meeting. 
    \item The operating budget shall be the major budget for the fiscal year and provide for all expenditures of the organization for the subsequent year.
    \item The operating budget shall be approved by a majority vote of the general members present and voting at a valid general meeting. 
    \item The banking business of the organization, or any part thereof, shall be transacted with such bank, trust company or other firm or body corporate as the Executive may designate, appoint or authorize from time to time and all such banking business, or any part thereof, shall be transacted on the organization's behalf by one or more Officers or other persons as the Executive may designate, direct or authorize from time to time and to the extent thereby provided.
    \item The President, the Vice-Presidents, and only in special circumstances a member appointed by the Executive Committee shall be the sole signing authorities of banking instruments for the organization.  
    \item \orgname will ensure that proper and accurate financial records are maintained and passed on to incoming executives following each year’s elections.
    \item	\orgname will accept full financial and production responsibility for all activities it sponsors, plans, or executes.
\end{enumerate}


% Article 7
\section{General Meetings}
\begin{enumerate}[{7}.1]
    \item The purpose of General Meetings is to provide a forum for executives to overview the activities of the organization and solicit feedback from members, to engage in policy-making, to propose amendments to the constitution, and to report on the financial status of the organization.
    \item General meetings will be facilitated by a Chairperson selected by the general membership from the executive committee. The Chairperson shall be responsible for:
    \begin{enumerate}[{7.2}.1]
        \item	Formulating and distributing an agenda for each meeting no later than two (2) days before the meeting;
        \item	Ensuring appropriate conduct and leading the meeting in an efficient, reasonable manner;
        \item	Moderating the discussion at meetings according to the agenda;
        \item	Suspending members from participating in meetings for constitutional or procedural violations. 
    \end{enumerate}
    \item The procedure at meetings of members shall be governed in accordance with the process outlined in Appendix A.
    \item There shall be a minimum of one (1) general meeting held each semester. The date of each subsequent general meeting will be confirmed at the preceding general meeting and will be reiterated to members via email a minimum of two (2) calendar days prior to the meeting.
    \item General meetings may be called to order by the President, through a petition by a petition signed by one (1) executive members, or by a petition signed by five (5) non-executive general members. 
    \item General meetings are open to registered members of the organization only. Quorum will first be established by the presence of a simple and clear majority of the executives. 
    \item For quorum to remain valid, the number of non-executive general members present at a general meeting must exceed the number of executives present at all times. 
    \item All executives are expected to make brief progress reports on their activities at every general meeting.
    \item Minutes of all general meetings must be recorded and maintained for reference purposes. 
    \item Members must contact the Chairperson a minimum of 48 hours before a general meeting to inform them of new business they wish to discuss. The Chairperson will then add the discussion item to the agenda.
    \item Each member of the organization shall be entitled to one (1) vote at a general meeting except the Chairperson who shall only vote in the event of a tie. 
    \item Any question at a valid general meeting shall be decided by a show of hands. 
    \item Whenever a vote by show of hands occurs, a declaration by the chairperson that the vote upon the question has been carried, carried by a particular majority, or failed shall be recorded in the minutes of the meeting.
    \item In case of an equality of votes at a valid general meeting, the Chairperson of the meeting shall have the deciding vote. 
    \item The Chairperson presiding over a meeting of members may, with the consent of the majority of members, decide to adjourn these meetings from time to time.
\end{enumerate}


% Article 8
\section{Executive Meetings}
\begin{enumerate}[{8}.1]
    \item The purpose of executive meetings is to provide a forum for the organization’s executives to discuss and make decisions on day-to-day matters affecting the organization. 
    \item Executive meetings will be facilitated by the President of the organization. The President shall be responsible for:
    \begin{enumerate}[{8.2}.1]
        \item	Formulating and distributing an agenda for each meeting;
        \item	Ensuring appropriate conduct and leading the meeting in an efficient, reasonable manner;
        \item	Moderating the discussion at meetings according to the agenda;
    \end{enumerate}
    \item There shall be a minimum of one (1) executive meeting held during the period September 1 to April 30. The date of each subsequent executive meeting will be confirmed at the preceding meeting and will be reiterated to executives via email a minimum of two (2) calendar days prior to the meeting.
    \item The frequency of executive meetings occurring between May 1 and August 31 will be left to the discretion of the executive committee. 
    \item Executive meetings may be called to order by the President or through a petition signed by one (1) executive members. 
    \item Executive meetings are restricted to executive members only. Quorum will be established by the presence of a simple and clear majority of the total executives for the organization.
    \item Minutes of all executive meetings must be recorded and maintained for reference purposes.
    \item Executives must notify the President a minimum of six (6) hours before an executive meeting to inform them of new business they wish to discuss. The President will then add the discussion item to the agenda.
    \item Each executive member of the organization shall be entitled to one (1) vote at a valid executive meeting. 
    \item	Any question at an Executive Meeting shall be decided by a show of hands. 
    \item	Whenever a vote by show of hands occurs, a declaration by the President that the vote has been carried, carried by a particular majority, or failed shall be recorded in the minutes of the meeting.
    \item	In case of an equality of votes at an Executive Meeting, the motion will be recorded as having failed. 
    \item	The President may, with the consent of the majority of executives, decide to adjourn these meetings from time to time.
\end{enumerate}


% Article 9
\section{Emergency Meetings}
\begin{enumerate}[{9}.1]
    \item	Emergency meetings can be called for extenuating or unforeseen circumstances that may arise from time to time. 
    \item	These meetings must abide the respective rules outlined in sections VII and VIII depending on the nature of the meeting.
    \item	Notice of these meetings must be provided a minimum of 24 hours in advance through email. 
    \item	Less notice for emergency meetings may be provided at the discretion of the President in agreement with a minimum of five (5) general members. 
\end{enumerate}


% Article 10
\section{Elections}
\begin{enumerate}[{10}.1]
    \item Executive elections will be held prior to March 31 each year.
    \item Candidates for executive positions shall be selected through an application process subject to meeting a set of minimum qualifications for holding a particular position. These qualifications will be established by the outgoing executive team each year prior to the commencement of the application submission period.  
    \item	Only student members who meet the minimum qualifications to hold an executive position shall be permitted to participate in an election and hold executive positions. 
    \item All screening of candidates will be conducted by a committee comprised of majority number of non-executive general members and minority number of executives who will assess each candidate’s qualifications against pre-established criteria for holding the positions.
    \item Notification of the acceptance of applications for executive positions will be sent via email to all general members a minimum of twenty-one (21) calendar days prior to the general meeting at which the election will be held. 
    \item All application periods must commence a minimum of fourteen (14) calendar days prior to the general meeting at which the election will be held. The application period must end a minimum of seven (7) days prior to the general meeting at which the election will be held. 
    \item Successful candidates will be permitted to give a short speech at the general meeting where the election is being held. Each speech will be followed by a short question and answer period. The length of each speech and the question/answer period will be left to the discretion of the Chairperson.
    \item Elections shall be conducted by secret ballot, and overseen by an election oversight committee separate and unique from the candidate selection committee described in section 10.4.  
    \item This committee will be comprised of two (2) non-executive general members and one (1) executive. 
    \item Successful candidates will be determined by accrual of the most number of votes tallied from amongst the general membership. 
    \item Final results of the election must be presented to the membership for ratification of the process only. The results themselves should not be brought into question; only the process through which these results were tabulated. 
    \item If an error in the process is found, the election should be re-held at the final General Meeting with a new election oversight committee.
    \item Candidates who run for a position unopposed must receive a simple and clear majority of the total eligible votes at a valid general meeting in which an election is held to be declared the winner of that election. 
\end{enumerate}


% Article 11
\section{Amendments}
\begin{enumerate}[{11}.1]
    \item The organization may make, amend or repeal the constitution or certain sections therein. 
    \item Notice of a meeting called to consider such a resolution shall be given as follows: 
        \begin{enumerate}[{11.2}.1]
            \item Notice of the full text of the proposed constitutional amendment shall be given to each member at least fourteen (14) days prior to the date of the meeting called to consider the change; 
            \item A summary of the rationale for the proposed amendment shall be given to each member at least fourteen (14) days prior to the date of the meeting called to consider the change.
        \end{enumerate}
    \item Amendments to the constitution require the approval of two-thirds of the members present at a valid general meeting (a general meeting that has achieved quorum).
    \item The general membership must have the final say on amendments to the constitution.  
\end{enumerate}


% Article 12
\section{Transition}
\begin{enumerate}[{12}.1]
    \item	All outgoing executives are required to transfer all organizational resources used relative to a particular role over the course of the preceding year to new executives upon leaving the position. 
    \item	All outgoing executives are responsible for providing a detailed report to incoming executives that stipulates the status of ongoing projects in their portfolio and evaluations of previous projects and programs that they lead.
    \item	All outgoing and incoming executives will participate in a joint training session occurring no later than the end of May each year to assist with the transition between new executive teams.
\end{enumerate}


% Article 13
\section{Emergency Powers}
\begin{enumerate}[{13}.1]
    \item	In the case of extenuating circumstances, the executive shall be afforded the ability to act without direction from the organization’s members. 
    \item	An extenuating circumstance is defined as any instance that may jeopardize the immediate functioning of the organization including but not limited to: executive vacancies, unexpected cancellations, removal from position, or lack of response from members.
    \item	Emergency powers may only be used for such a period of time as is needed to address an extenuating circumstance. 
    \item	General members have the ability to remove emergency powers where appropriate through submission of a signed petition from at least 10% of the entire general membership. 
\end{enumerate}


% Article 14
\section{Food Handling on Campus}
\begin{enumerate}[{14}.1]
    \item \orgname will conform to Provincial and Municipal Health Regulations when events which include the sale and/or service of food products are held on the University of Toronto campus.
\end{enumerate}


% Article 15
\section{Precedence of University Policies}
\begin{enumerate}[{15}.1]
    \item \orgname will abide by all pertinent University of Toronto policies, procedures, and guidelines. Where the University’s policies, procedures, and guidelines conflict with those of \orgname, the University’s policies, procedures, and guidelines will take precedent.
\end{enumerate}


% Article 16
\section{Legal Liability}
\begin{enumerate}[{16}.1]
    \item The University of Toronto does not endorse the \orgname’s beliefs or philosophy nor does it assume legal liability for the group’s activities on or off campus.\end{enumerate}


% Article 17
\section{Banking}
\begin{enumerate}[{17}.1]
    \item \orgname agrees to provide the name of the bank, the branch number and address, transit number, bank account number, and a list of all signing officers for all bank accounts opened in the organization’s name to the Department of Student Life, University of Toronto.
\end{enumerate}


%%%%%%%%%%%%%%%%%%%% APPENDIX %%%%%%%%%%%%%%%%%%%%

\pagebreak[4]
\appendix
\part{Appendices}

% Appendix A
\section{General Meeting Rules of Order}


% Appendix A1
\subsection{Call to Order}
\begin{enumerate}[{A.1}.1]
    \item The Chairperson may call the meeting to order only if a quorum of executives and non-executive general members is present in person. If a quorum does not exist, the meeting is not qualified to conduct business. A general member may not appear by proxy or mail ballot.
    \item The meeting must be open to all applicable general members. General members must receive notice of the meeting in accordance with, the constitution.
\end{enumerate}


% Appendix A2
\subsection{Review of the Agenda}
\begin{enumerate}[{A.2}.1]
    \item The first draft of the agenda is prepared by the chairperson prior to the meeting. Agenda items should ordinarily appear in the order set forth in these rules of order. 
    \item The agenda belongs to all general members. The agenda may be modified only by a majority vote. This power should only be used when necessary as proper functioning of meetings and the organization requires advance planning. 
    \item At this point in the agenda, general members may add or delete items from the agenda and may change the order of presentation.
    \item When possible, changes to the agenda should be done by acquiescence of all general members. Formal voting on the agenda is only necessary where it Lwappears to the chairperson that there is a disagreement.
\end{enumerate}


% Appendix A3
\subsection{Approval of Previous Minutes}
\begin{enumerate}[{A.3}.1]
    \item The minutes need not be read aloud but they should be entered into the organization’s official minute ledger upon approval by the general membership.
    \item The minutes are prepared by either the secretary or some other individual appointed by the general membership to act as recording secretary. Any general member may suggest changes to the minutes before the general membership adopts them. The suggested changes should be set forth in the minutes for the record, and then the general membership should adopt or reject such changes.
    \item Minutes should state precisely each motion considered by the general membership, and identify the general members voting in favor, against, or abstaining, and whether the motion was carried. Minutes need not reflect the comments made except in those instances when the member desires to make his/her comments recorded. 
    \item When possible, changes to the minutes and adoption of the minutes should be done by acquiescence of all general members. Formal voting on the minutes is only necessary where it appears to the Chairperson that there is a disagreement.
\end{enumerate}


% Appendix A4
\subsection{Executive Reports}
\begin{enumerate}[{A.4}.1]
    \item Executives may report their findings or recommendations to the general membership at this point of the agenda. 
    \item The full report should be presented and then general members, in turn, may ask questions or comment. It is not appropriate to make motions or discuss items of business during this portion of the meeting.
    \item This time should also be used for any presentations to be made to the general membership.
\end{enumerate}


% Appendix A5
\subsection{Open Forum}
\begin{enumerate}[{A.5}.1]
    \item It is the custom and practice of most organizations to allow general members an open forum to ask questions and speak about their concerns to an executive after a report has been provided.
    \item Strict time limitations should be imposed by the Chairperson and these limitations must be enforced. Each general member should address the Chairperson regarding an issue and must speak courteously and to the point.
\end{enumerate}


% Appendix A6
\subsection{Old and New Business}
\begin{enumerate}[{A.6}.1]
    \item	All items that were tabled during previous meetings must be revisited during the business portion of the agenda occurring after executive reports. 
    \item	The general membership may vote to postpone consideration of any old business or it may remove any item from consideration.
    \item	Except in the case of emergency business, all new items of business are heard only after all of the old items have been addressed by the general membership. 
    \item	All business must be conducted in the form of motions or resolutions adopted by a vote of the general membership.
\end{enumerate}


% Appendix A7
\subsection{Motions and Deliberations}
\begin{enumerate}[{A.7}.1]
    \item	When an item of business is to be discussed, the Chairperson announces the item to be discussed and opens the floor to discussion. 
    \item	No general member may speak until recognized by the Chairperson. No general member may interrupt the speaker who has the floor.
    \item	The Chairperson may impose reasonable time limitations. All time limitations must be uniformly imposed upon all of the general members. The speaker shall be given a one-minute warning before time runs out.  By vote of a majority of the general membership, time limits may be extended.
    \item	The Chairperson is to recognize each general member in turn. Discussion shall be limited to the item of business at hand, and the Chairperson shall have the authority to take the floor from a speaker who does not limit discussion to the item of business at hand.  
    \item	No general member may speak to an issue for a second time until all other general members have had the opportunity to speak to it for the first time. Likewise, no general member may speak to an issue for a third time until all other general members have had the opportunity to speak to it for a second time.
    \item	When it appears to the Chairperson that all general members have had the opportunity to fully discuss the matter at hand, the Chair should announce that the item of business is ready for a vote.
\end{enumerate}


% Appendix A8
\subsection{Voting}
\begin{enumerate}[{A.8}.1]
    \item There are 3 basic motions for each item of business:
    \begin{itemize}
        \item	A motion to adopt a specific action by the board.
        \item	A motion to postpone the item to another meeting (including fact-finding assignments to a person or committee).
        \item	A motion to remove an item from consideration 
    \end{itemize}
    \item The general membership is limited to discussing one item of business at a time, but there are no limits to the number of motions that may be considered as to how to dispose of that item of business. 
    \item After the general membership has had the opportunity to discuss each motion presented for consideration, the Chairperson will call each motion presented to a vote. 
    \item The fact that a motion has been adopted or failed does not prevent the item of business from being added to the agenda in the future and all motions may be reconsidered at any time by the general membership. 
\end{enumerate}

\begin{center}
\vfill
{\sc This is the last page.}
\end{center}
\end{document}
